\documentclass[main.tex]{subfiles}
\begin{document}

\section*{Topologie des \( \R^d \)}

\begin{karte}{Norm auf einen Vektorraum}
    Eine Norm auf einen reellen oder komplexen Vektorraum
    ist eine Funktion: \( \abb{\norm{\cdot}}{X}{\R} \) mit
    \begin{description}
        \item[Positivität] \( \norm{x} \geq 0 \; \forall x\in X \) 
        und \( \norm{x} = 0 \Leftrightarrow x = 0 \).
        \item[Homogenität] \( \norm{\lambda x} = \abs{\lambda} \norm{x}
        \; \forall \lambda \in \R (\C), x\in X \).
        \item[\( \Delta \)-Ungleichung] \( \norm{x+y}
        \leq \norm{x} + \norm{y} \; \forall x,y \in X \)
    \end{description}
\end{karte}

\begin{karte}{Metrik und metrischer Raum}
    Menge \(X\) mit einer Funktion \( \abb{d}{X\times X}{\R} \)
    mit den Eigenschaften
    \begin{description}
        \item[Positivität] \( d(x,y) \geq 0, d(x,y) = 0 \Leftrightarrow x=y
        \; \forall x,y \in X \)       
        \item[Symmetrie] \( d(x,y) = d(y,x) \; \forall x,y \in X \)
        \item[\( \Delta \)-Ungleichung] 
        \( d(x,y) \leq d(x,z) + d(z, y) 
        \; \forall x,y,z \in X \).
    \end{description}
    \(d\) heißt Metrik und \( (X, d) \) heißt metrischer 
    Raum.
\end{karte}

\begin{karte}{Offene Kugel}
    Sei \(X\) ein metrischer Raum mit Metrik \(d\).
    \[ x_0 \in X, r > 0 \quad B_r(x_0) 
    := \set{ x\in X: d(x, x_0) < r } \] 
    ist die offene Kugel mit Radius \(r\) um \(x_0\).
    \[ B_r(x_0) = \set{x \in \R^d: 
    \norm{x - x_0}_2 < r}. \]
\end{karte}

\begin{karte}{Offene Mengen}
    Sei \( (X,d) \) ein metrischer Raum. Eine Menge 
    \( U \subset X \) heißt offen, falls 
    \[ \forall x\in U \; \exists r = r_x > 0 \]
    mit 
    \[ B_r(x) \subset U. \]
\end{karte}

\begin{karte}{Topologie}
    Man nehme metrischen Raum \( (X, d) \).
    Die Menge aller offenen Teilmengen von \(X\) 
    ist eine Topologie, das heißt
    \begin{enumerate}
        \item \( \emptyset, X \) sind offen.
        \item Der Durchschnitt endlich vieler
        offener Mengen ist offen.
        \item Die Vereinigung beliebig vieler offener
        Mengen ist offen. 
    \end{enumerate}
\end{karte}

\begin{karte}{Inneres, Abschluss, Rand von Mengen}
    Sei \( (X, d) \) ein metrischer Raum, \( M \subset X \).
    Dann ist 
    \[ M^o 
    := \set{x \in M: \exists r_x > 0: B_r(x)\subset M}\]
    \[= \set{x \in M: \exists \varepsilon > 0: 
    B_\varepsilon(x) \subset M }.\]
    \( M^o \) ist offen und \( U \) offen 
    und \( U \subset M \Rightarrow U \subset M^o \).
    (d. h. \( M^o \) ist die größte offene Teilmenge 
    von \(M\))\\
    Abschluss:
    \[ \overline{M} := 
    \set{x \in X: \; \forall \varepsilon > 0: 
    B_\varepsilon(x) \cap M \neq \emptyset} \]
    \( \overline{M} \) ist abgeschlossen und ist 
    \(A \supset M \), \( A \) abgeschlossen 
    \( \Rightarrow A \supset \overline{M}. \)
    Rand: 
    \[ \delta M := \set{ x \in X: \; \forall \varepsilon > 0: 
    B_\varepsilon(x) \cap M \neq \emptyset, 
    B_\varepsilon(x) \cap M^C \neq \emptyset } = \overline{M} \setminus M^o\]
    \[ M^C = X \setminus M. \]
    \[ M^o \subset M \subset \overline{M}. \]
\end{karte}

\begin{karte}{Hausdorffsche Trennungseigenschaft}
    Sei \( X \) ein metrischer Raum, 
    \( x,y \in X, x\neq y \). 
    \[ \Rightarrow \exists \varepsilon > 0: 
    B_\varepsilon(x) \cap B_\varepsilon(y) = \emptyset. \]
\end{karte}

\begin{karte}{Konvergenz}
    Sei \( X \) ein metrischer Raum. Die Folge 
    \( (x_n)_n \) von Punkten \( x_n \in X \; \forall n\in\N \)
    (schreiben \( (x_n)_n \subset X \)) konvergiert 
    gegen \( x \in X \), falls 
    \begin{align*}
        &\forall \varepsilon > 0 \; \exists K \in \N: 
        d(x_n, x) < \varepsilon \;\forall n \geq K. \\
        &\Leftrightarrow \; \forall \varepsilon > 0 
        \; \exists K \in \N: x_n \in B_\varepsilon(x) 
        \; \forall n \geq K 
    \end{align*}
    \( \Leftrightarrow \forall \varepsilon > 0  \) 
    ist \( x_n \in B_\varepsilon(x) \) für fast alle 
    \( n\in\N \)
    \[ \Leftrightarrow \limes{n} d(x_n, x) = 0 \]
    Schreiben \( x_n \rightarrow x \), oder 
    \( \limes{n} x_n = x \).
\end{karte}

\begin{karte}{Abgeschlossene Menge}
    Sei \( X \) ein metrischer Raum und \( A \subset X \) 
    heißt abgeschlossen, falls 
    \[ (x_n)_n \subset A, x_n \rightarrow x 
    \Rightarrow x \in A. \]
    In einem metrischen Raum 
    \( X \) gilt für alle \( A \subset X \):
    \[ A \text{ ist offen } 
    \Leftrightarrow A^C = X \setminus A \text{ ist 
    abgeschlossen}. \]
\end{karte}

\begin{karte}{Eigenschaften abgeschlossener Mengen}
    Für Teilmengen eines metrischen Raums \(X\) gilt: 
    \begin{enumerate}
        \item \( \emptyset, X \) abgeschlossen.
        \item Die Vereinigung endlich vieler abgeschlossener 
        Mengen ist abgeschlossen.
        \item Der Durchschnitt beliebig vieler abgeschlossener 
        Mengen ist abgeschlossen.
    \end{enumerate}
\end{karte}

\begin{karte}{Häufungspunkte, isolierte Punkte}
    Sei \(X\) metrischer Raum, \( M \subset X \) \\
    Ein Punkt \( x \in X \) heißt Häufungspunkt von \(M\),
    falls 
    \[ \forall \varepsilon > 0: 
    B_\varepsilon(x) \cap M \setminus \set{x} 
    \neq \emptyset. \]
    Ein Punkt \( x \in X \) heißt isolierter Punkt 
    von \( M \), falls 
    \[ \exists \varepsilon > 0: B_\varepsilon(x) \cap M 
    = \set{x}. \]
\end{karte}

\begin{karte}{Dichte Menge}
    Eine Teilmenge \(M\) eines metrischen Raumes heißt
    dicht, falls \(\overline{M} = X\).
\end{karte}

\begin{karte}{Cauchyfolge}
    Sei \( X \) ein metrischer Raum. 
    Eine Folge \( (x_n)_n \subset X \) heißt 
    Cauchyfolge, falls
    \[ \forall \varepsilon > 0 \; \exists K \in \N: 
    d(x_n, x_m) < \varepsilon \; \forall n,m \geq K \]
    \[ \Leftrightarrow \limessup{n} \limessup{m} 
    d(x_n, x_m) = 0. \]
    Ein metrischer Raum heißt vollständig, falls jede 
    Cauchyfolge konvergiert.
\end{karte}

\begin{karte}{Äquivalenz von Normen}
    \( \norm{\cdot}_a, \norm{\cdot}_b \) sind äquivalent, wenn
    \[ \exists c_1, c_2: c_1\norm{x}_a \leq \norm{x}_b 
    \leq c_2\norm{x}_a \; \forall x. \]
\end{karte}

\begin{karte}{Stetigkeit}
    Seien \(X, Y \) metrische Räume, \(\abb{f}{X}{Y}\)
    heißt stetig in \(x_0 \in X\), falls 
    \[ \forall \varepsilon > 0 \; \exists \delta > 0:
    d_Y(f(x), f(x_0)) < \varepsilon \; \forall x \in X: 
    d_X(x, x_0) < \delta \]
    oder äquivalent 
    \[ f(B_\delta(x_0)) \subset B_\varepsilon(f(x_0)) \]
    \(f\) stetig auf \(X\) falls \(f\) in jedem Punkt 
    \(x_0 \in X\) stetig ist.
\end{karte}

\begin{karte}{Lipschitzstetigkeit}
    \( \abb{f}{X}{Y} \) heißt Lipschitzstetig 
    mit Konstante \( L \geq 0 \), falls 
    \[ d(f(x_1), f(x_2)) 
    \leq L d(x_1, x_2) \;\forall x_1, x_2 \in X. \]
\end{karte}

\begin{karte}{Folgenkriterium für Stetigkeit}
    Sei \( \abb{f}{X}{Y} \) stetig in \(x_0\)\\
    \( \Leftrightarrow \) für jede Folge 
    \( (x_n)_n \subset X, x_n \rightarrow x_0 \) 
    folgt \( f(x_n) \rightarrow f(x_0) \).
\end{karte}

\begin{karte}{Charakterisierung von Stetigkeit}
    Seien \( X, Y \) metrische Räume. Dann gilt \( \abb{f}{X}{Y} \)
    ist stetig \( \Leftrightarrow \) Urbilder offener Mengen sind offen:
    \( \forall V \subset Y \) offen \(\Rightarrow f^{-1}(V) \) ist offen in \( X \).
\end{karte}

\begin{karte}{Folgenkompaktheit}
    Ein metrischer Raum \( X \) ist folgenkompakt oder kurz kompakt, 
    falls jede Folge \((x_n)_n \subset X \) eine Teilfolge
    \( (x_{n_k})_k \subset X \) hat, die gegen ein \( x \in X \) konvergiert. \\
    Eine Menge \( K \subset \R^d \) (mit Euklidischen Abstand sogar jede
    Norm) ist kompakt \( \Leftrightarrow K \) ist abgeschlossen und
    beschränkt.
\end{karte}

\begin{karte}{Totale Beschränktheit}
    \(M\) ist total beschränkt: 
    \[ \forall \varepsilon > 0 \; 
    \exists \text{ endlich viele } x_1,\dots,x_n \in X:
    M \subset \bigcup_{j=1}^n B_\varepsilon(x_j). \]
\end{karte}

\begin{karte}{Bilder kompakter Mengen}
    Sei \( \abb{f}{X}{Y} \) stetig, \( X, Y \) metrische Räume
    und \(X\) sei kompakt
    \( \Rightarrow \)
    \begin{enumerate}
        \item \( f(X) \subset Y \) ist kompakt in \( Y \).
        \item Ist \(f\) injektiv, so ist \( \abb{f^{-1}}{f(X)}{X} \)
        stetig.
    \end{enumerate}
\end{karte}

\begin{karte}{Extrema in kompakten Räumen}
    Sei \( \abb{f}{X}{\R} \) stetig, \( X \) kompakter 
    metrischer Raum.\\
    \( \Rightarrow f \) nimmt sein Minimum und 
    Maximum an.
\end{karte}

\begin{karte}{Gleichmäßige Stetigkeit in kompakten Räumen}
    Sei \( \abb{f}{X}{Y} \) stetig, 
    \( X, Y \) metrische Räume, \(X\)
    kompakt. \\
    \( \Rightarrow f \) ist sogar gleichmäßig 
    stetig:
    \[ \forall \varepsilon > 0\; \exists 
    \delta(\varepsilon) = \delta > 0:
    x_1, x_2 \in X: d(x_1, x_2) < \delta
    \Rightarrow d(f(x_1), f(x_2)) < \varepsilon. \]
\end{karte}

\end{document}
