\documentclass[main.tex]{subfiles}
\begin{document}

\section*{Banachscher Fixpunktsatz}

\begin{karte}{\(\alpha\)-Kontraktion}
    Sei \( M \) ein metrischer Raum mit Metrik \(d\). \( \abb{T}{M}{N} \) ist 
    eine \( \alpha \)-Kontraktion, falls \( d(T(x), T(y)) \leq \alpha d(x,y) 
    \;\forall x,y\in M \).
\end{karte}

\begin{karte}{Banachscher Fixpunktsatz}
    Sei \( M \) ein vollständiger metrischer Raum, \( \abb{T}{M}{M} \) 
    eine \( \alpha \)-Kontraktion (\( 0 \leq \alpha < 1 \)). Dann gibt 
    es genau einen Fixpunkt \( x^* \) und man nehme irgendein \( x \in M \) 
    \[ \Rightarrow x^* = \limes{n} T^n(x). \]
    mit \( T^1(x) = T(x), T^{n+1}(x) = T(T^n(x)) \)
    und a priori Fehlerabschätzung: 
    \[ d(x^*, T^n(x)) \leq \frac{1}{1-\alpha} \alpha^n d(x, T(x)). \]
\end{karte}

\begin{karte}{Vereinfachtes Newton-Verfahren}
    Genügt die Funktion \(T\) einer Lipschitzbedingung
    in einer offenen Kugel \( B_r(a) \) mit 
    Lipschitzkonstante \( \alpha = \frac{1}{2} \) 
    und ist \(\abs{A^{-1} f(a)} < \frac{r}{2}\), so hat \(f\)
    in \(B_r(a)\) genau eine Nullstelle \(\xi\).
\end{karte}

\end{document}