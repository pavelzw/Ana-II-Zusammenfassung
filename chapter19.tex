\documentclass[main.tex]{subfiles}
\begin{document}

\section*{Kurvenintegrale}

\begin{karte}{\( \mathcal{C}^1 \)-Kurve}
    Ist \( I \subset \R \) Intervall und
    \( \gamma \in \mathcal{C}^1(I, \R^d) \),
    so heißt \(\gamma \ \mathcal{C}^1\)-Kurve.
\end{karte}

\begin{karte}{Bogenlänge}
    Sei \(\abb{\gamma}{[a,b]}{\R^d}\) eine \(\mathcal{C}^1\)-Kurve
    \[ L(\gamma) := \integralx{\abs{\gamma'(t)}}{a}{b}{t} \]
    Bogenlänge (Länge der Kurve).\\
    Zerlegung \(a = t_0 < t_1 < \ldots < t_N = b\)
    \[ \sum_{j=1}^N \abs{\gamma(t_j) - \gamma(t_{j-1})} 
    = \sum_{j=1}^N \abs{\gamma'(t_j)} (t_j - t_{j-1}) 
    \approx \integralx{\abs{\gamma'(t)}}{a}{b}{t}. \]
\end{karte}

\begin{karte}{Umparametrisierung}
    Seien \(\abb{\gamma_1, \gamma_2}{I_j}{\R^d}\) zwei \(\mathcal{C}^1\)-Kurven
    Dann heißt \(\gamma_2\) Umparametrisierung von \( \gamma_1 \),
    falls es eine Bijektion \(\varphi \in \mathcal{C}^1(I_2, I_1)\)
    mit \(\varphi' \neq 0 \) existiert mit 
    \[ \gamma_2 = \gamma_1 \circ \varphi. \]
    Die Bijektion \( \varphi \) heißt Parametertransformation.
\end{karte}

\begin{karte}{Bogenlänge von Umparametrisierungen}
    Seien \(\gamma_j \in \mathcal{C}^1(I_j, \R^d), 
    j \in \set{1,2}\). Wenn \(\gamma_2\) eine Umparametrisierung  
    von \( \gamma_1 \) ist, so ist 
    \[ L(\gamma_1) = L(\gamma_2). \]
\end{karte}

\begin{karte}{Bogenlängenparametrisierung}
    Eine Kurve \(c \in \mathcal{C}^1(I, \R^d)\) heißt nach Bogenlänge
    parametrisiert, falls
    \[ \abs{c'(s)} = 1 \; \forall s \in I. \]
    Sei \( [a,b] \subset I \).
    \[ L(c \vert_{[a,b]}) = \integralx{\abs{c'(s)}}{a}{b}{s}
    = b - a. \]
    D. h. \(c\) bildet \([a,b]\) längentreu nach \(\R^d\) ab.\\
    Sei \( \gamma \in \mathcal{C}^1([a,b],\R^d) \) eine Kurve 
    mit Länge \(L = L(\gamma)\) und \(\gamma'(t) \neq 0 
    \; \forall t \in [a,b] \) (\( \gamma \) ist regulär).
    Dann gibt es eine \(\mathcal{C}^1\)-Bijektion \(\varphi\) 
    mit \( \varphi' > 0 \) so, dass \(c = \gamma \circ \varphi\) nach 
    Bogenlängeparametrisierung ist.
\end{karte}

\begin{karte}{Stückweise \( \mathcal{C}^1 \)-Kurve}
    \(\gamma \in \mathcal{C}([a,b], \R^d)\) heißt stückweise
    \(\mathcal{C}^1\)-Kurve, falls eine Zerlegung
    \(a = t_1 < t_1 < \cdots < t_N = b\) existiert so, dass
    für \( I_k = [t_{k-1}, t_k]\)
    \[ \gamma \vert_{I_k} \in \mathcal{C}^1(I_k, \R^d) 
    \; \forall k = 1,\ldots,N. \]
    \( \gamma \in P \mathcal{C}^1([a,b], \R^d)\).
\end{karte}

\begin{karte}{Linienintegral}
    Ist \( F \in \mathcal{C}(U, \R^d) \), \(U \subset \R^d\) offen. 
    \( \gamma \in P\mathcal{C}^1([a,b], U) \), so heißt 
    \[ \int_\gamma F\cdot d \vec{x} 
    := \integralx{\scalarprod{ F(\gamma(t)) }{ \gamma'(t) } }{a}{b}{t} \]
    Linienintegral.
\end{karte}

\begin{karte}{Eigenschaften des Kurvenintegrals}
    a): linear: \\
    \( F_j \in \mathcal{C}(U, \R^d), \lambda_j \in \R, j = 1,2 \)
    \[ \gamma \in P\mathcal{C}^1([a,b], U) \Rightarrow 
    \int_\gamma (\lambda_1 F_1 + \lambda_2 F_2)\; d\vec{x} 
    = \lambda_1 \int_\gamma F_1\; d\vec{x} + \lambda_2 \int_\gamma F_2
    \; d\vec{x}. \]
    b): Additivität bei Zerlegung:\\
    Ist \(\gamma \in P\mathcal{C}^1([a,b], U), 
    F \in \mathcal{C}^1(U, \R^d) \)
    und \(a = t_0 < \ldots < t_N = b\) Zerlegung von \([a,b]\)
    \[\gamma_k := \gamma\vert_{[t_{k-1}, t_k]}
    \Rightarrow \integralx{F}{\gamma}{}{\vec{x}}
    = \sum_{k=1}^N\integralx{F}{\gamma_k}{}{\vec{x}}.\]    
    c) Invarianz unter Umparametrisierung:\\
    Sind \(\gamma \in P\mathcal{C}^1(I,\R^d), \varphi \in 
    \mathcal{C}^1(I_2, I_1)\)
    eine Parametertransformation, so gilt, je nach Vorzeichen 
    von \(\varphi'\)
    \[ \int_{\gamma \circ \varphi} F \; d\vec{x} 
    = \pm \int_\gamma F \; d\vec{x} \]
    \(+: \varphi' > 0, \\
    -: \varphi' < 0\).
\end{karte}

\begin{karte}{Abschätzung Kurvenintegral}
    Sei \(U \subset \R^d\) offen, \(F \in \mathcal{C}^1(U, \R^d), \gamma \in P\mathcal{C}^1([a,b], U)\)
    \[ \abs{\int_\gamma F d\vec{x}} \leq \norm{F \circ \gamma}_{\infty,[a,b]} L(\gamma) 
    = \underset{t \in [a,b]}{\sup} \abs{F(\gamma(t))} L(\gamma).\]
\end{karte}

\begin{karte}{Gradientenfeld}
    Sei \(U \subset \R^d\) offen. Das Vektorfeld \(F \in \mathcal{C}(U,\R^d)\)
    heißt Gradientenfeld (bzw. Konservativ),
    falls ein \(\varphi \in \mathcal{C}^1(U, \R)\) mit 
    \[ \nabla \varphi = F. \] 
    \(\varphi\) heißt Stammfunktion oder Potential von \(F\). 
    \( F(x) = \nabla \varphi (x) \).
\end{karte}

\begin{karte}{Stammfunktion Eindeutigkeit}
    Ist \( U \subset \R^d \) wegezusammenhängend, so 
    ist eine Stammfunktion \( \varphi \) von \\
    \( F \in \mathcal{C}(U, \R) \) 
    bis auf eine additive Konstante eindeutig bestimmt.
\end{karte}

\begin{karte}{Geschlossene Kurve}
    Eine Kurve \(\abb{\gamma}{[a,b]}{\R^d}\)
    heißt geschlossen, falls \(\gamma(a) = \gamma(b)\).
\end{karte}

\begin{karte}{Wegunabhängigkeit der Kurvenintegrale}
    Sei \( U \subset \R^d\) offen und wegweise zusammenhängend.
    Für \( F \in \mathcal{C}(U, \R^d) \) sind äquivalent:
    \begin{enumerate}
        \item \( F \) ist Gradientenfeld.
        \item Sei \(\gamma \in P\mathcal{C}^1([a,b], U)\) und 
        \(\gamma\) geschlossen. Dann gilt 
        \[ \int_\gamma F d\vec{x} = 0. \]
        \item  Für je zwei Kurven \(\gamma_1, \gamma_2 \in P\mathcal{C}^1([a,b],U)\)
        mit \(\gamma_1(a) = \gamma_2(a), \gamma_1(b) = \gamma_2(b)\) ist    
        \[ \int_{\gamma_1} F d\vec{x} = \int_{\gamma_2} F d\vec{x}. \]
    \end{enumerate}
\end{karte}

\begin{karte}{Rotationsfreiheit}
    Für \( U \subset \R^d \) ist \( F \in \mathcal{C}(U, \R^d) \)
    ein Gradientenfeld.
    \[ \Rightarrow \partial_l F_j 
    = \partial_j F_l \;\forall 1 \leq l,j \leq d. \]
\end{karte}

\begin{karte}{Homotopie}
    Sei \(U \subset \R^d\) offen. Eine Homotopie in \(U\)
    zwischen Kurven \(\gamma_1, \gamma_2 \in \mathcal{C}([a,b], U) \)
    ist eine Abbildung \(\gamma \in \mathcal{C}([a,b]\times [0,1], U)\) mit
    \[ \gamma(\cdot, 0) = \gamma_1, \gamma(\cdot, 1) = \gamma_2. \]
    Gilt \( \gamma_1(a) = \gamma_2(a) = p, 
    \gamma_1(b) = \gamma_2(b) = q \) und gibt es eine Homotopie mit    
    \(\gamma(a,t) = p \; \forall t \in [0,1]\) mit 
    \(\gamma(b,t) = q \; \forall t \in [0,1]\), so heißen \(\gamma_1, \gamma_2\)
    homotop mit festen Endpunkten. Sind \( \gamma_1, \gamma_2 \)
    geschlossen und gibt es eine Homotopie mit 
    \(\gamma(a, t) = \gamma(b,t) \; \forall t \in [0,1]\),
    so heißen \(\gamma_1, \gamma_2\) geschlossen homotop in \(U\).
\end{karte}

\begin{karte}{Homotopie Identität}
    Sei \( U \subset \R^d, F \in \mathcal{C}^2(U, \R^d), 
    \gamma \in \mathcal{C}^1([a,b] \times [0,1], U) \) und 
    \( \partial_s \partial_t \gamma \in \mathcal{C}([a,b] \times [0,1], U) \).
    \begin{align*}
        &\int_{\gamma(\cdot, 1)} F \; d \vec{x}
        - \int_{\gamma(\cdot, 0)} F \; d\vec{x} \\
        = &\int_{\gamma(b, \cdot)} F \; d \vec{x} 
        - \int_{\gamma(a, \cdot)} F \; d\vec{x} \\
        + &\int_0^1 \int_a^b (\scalarprod{Df \circ \gamma
        \left[\ddxpartialboth{\gamma}{t}\right] }{ \ddxpartialboth{\gamma}{s} } 
        - \scalarprod{ DF \circ \gamma 
        \left[\ddxpartialboth{\gamma}{s}\right] }{ \ddxpartialboth{\gamma}{t} } ) 
        \; ds\; dt.
    \end{align*}
\end{karte}

\begin{karte}{Fundamentalsatz der Algebra}
    Jedes komplexe Polynom vom Grad \( n \geq 1 \) hat 
    mindestens eine komplexe Nullstelle.
\end{karte}

\begin{karte}{Affine Homotopie}
    Sei \(U \subset \R^d \) offen, \( F \in \mathcal{C}^1(U, \R^d), 
    \partial_j F_l = \partial_l F_j \).\\
    Kennen \( \gamma_0, \gamma_1 \in P\mathcal{C}^1([a,b], U) \).
    \(\gamma \) ist eine affine Homotopie mit 
    \[ \abb{\gamma}{[a,b] \times [0,1]}{\R^n}, \gamma(s,t) 
    := (1-t)\gamma_0(s) + t \gamma_1(s) \]
    Haben \(\gamma_0,\gamma_1\) dieselben 
    Endpunkte oder sind geschlossen und 
    \(\gamma([a,b]\times[0,1]) \subset U \)
    \[ \Rightarrow \int_{\gamma_0} F d\vec{x} 
    = \int_{\gamma_1} F d\vec{x}. \]
\end{karte}

\begin{karte}{Homotopieinvarianz}
    Sei \( U \subset \R^n \) offen, 
    \( F \in \mathcal{C}^1(U, \R^d), 
    \partial_j F_l = \partial_l F_j, 
    \gamma_0, \gamma_1 \in P\mathcal{C}^1([a,b], U) \)
    Homotopie in \( U \) mit festen 
    Endpunkten (oder geschlossene Homotopie)
    \[ \Rightarrow \int_{\gamma_0} F \; d\vec{x} 
    = \int_{\gamma_1} F \; d\vec{x}. \]
\end{karte}

\begin{karte}{Einfach zusammenhängend}
    Eine Menge \(U \subset \R^d\) heißt einfach zusammenhängend, wenn jede
    geschlossene Kurve \(\gamma \in \mathcal{C}([a,b], U)\) in \(U\)
    geschlossen homotop zu einer konstanten Kurve ist.
\end{karte}

\begin{karte}{Stammfunktionen von Vektorfeldern}
    Sei \( U \subset \R^d \) offen und einfach zusammenhängend. 
    Dann sind für ein Vektorfeld 
    \( F \in \mathcal{C}^1(U, \R^d) \) äquivalent:
    \begin{enumerate}
        \item \( \partial_j F_l = \partial_l F_j,\; l,j = 1,\ldots, d \).
        \item \( F \) hat eine Stammfunktion.
    \end{enumerate}
\end{karte}

\end{document}