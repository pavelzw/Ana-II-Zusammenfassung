\documentclass[main.tex]{subfiles}
\begin{document}

\section*{Taylor}

\begin{karte}{Taylorformel mit Integralrestglied}
    Für alle \( a \) und \( x\in I \)
    gilt 
    \[ f(x) = T_n f(x;a) + R_n f(x;a) \]
    \[ = \sum_{k=0}^n \frac{f^{(k)}(a)}{k!} (x-a)^k 
    + \int_a^x \frac{(x-t)^n}{n!} f^{(n+1)}(t) \; dt. \]
\end{karte}

\begin{karte}{Multiindexnotation}
    \[ \N_0^n = \N_0 \times \cdots \times \N_0 \]
    \[ \alpha \in \N_0^n, 
    \alpha = (\alpha_1, \ldots, \alpha_n) \]
    \[ \abs{\alpha} := \sum_{j=1}^n \alpha_j \]
    \[ \alpha! := \alpha_1! \cdots \alpha_n! \]
    \[ h\in \R^n: h^\alpha 
    := h^{\alpha_1}\cdots h^{\alpha_n} \]
    \[ \partial^\alpha := \partial_1^{\alpha_1} \cdots \partial_n^{\alpha_n} \]
    \[ h_j^0 = 1, \partial_j^0 = 1 \]
    Anzahl der Multiindizes mit \( \abs{\alpha} = l \) ist \( \binom{l+n-1}{n-1} \).
\end{karte}

\begin{karte}{Taylor in Multiindexnotation}
    Sei \( U \subset \R^n \) offen, \( f \in \mathcal{C}^k(U), 
    x \in U, [x,x+h] \subset U, h \in \R^n \).
    \[ \Rightarrow f(x + h) = \sum_{\substack{\alpha \in \N_0^n \\ \abs{\alpha} \leq k}} 
    \frac{ \partial^\alpha f(x) }{\alpha!} h^\alpha 
    + R_k(f, x)(h) \]
    mit 
    \[ R_k(f,x)(h) := \integralx{\frac{(1-t)^k}{(k-1)!} 
    \sum_{\abs{\alpha} = k} 
    (\partial^\alpha f(x+th)-\partial^\alpha f(x))h^\alpha}{0}{1}{t} \]
\end{karte}

\end{document}