\documentclass[main.tex]{subfiles}
\begin{document}

\section*{Uneigentliche Integrale und Reihen}

\begin{karte}{Uneigentliches Integral Definition}
    Sei \(f\) eine Regelfunktion auf dem Intervall \(I\) 
    mit Randpunkten \\
    \( a, b, -\infty \leq a < b \leq \infty \).
    \begin{enumerate}
        \item Ist \( I=[a,b) \), so definiert man im 
        Fall der Konvergenz:
        \[ \integral{f(x)}{a}{b} 
        := \limesx{\beta}{b-} \integral{f(x)}{a}{\beta} \]
        In diesem Fall heißt das uneigentliche Integral 
        \( \integral{f(x)}{a}{b} \) \\
        (genauer \( \underset{[a,b)}{\int} f(x) \dx \)) konvergent 
        und der Grenzwert dessen Wert.
        \item Ist \( I =(a,b) \) so definiert man
        \[ \integral{f(x)}{a}{b} 
        := \integral{f(x)}{a}{c} 
        + \integral{f(x)}{c}{b}, \]
        falls die uneigentlichen Integrale auf der rechten
        Seite für ein \(a < c < b\) existieren (und damit
        für alle \( a < c < b \)) und denselben Wert haben.
        \item Ein uneigentliches Integral über \(f\) 
        heißt absolut konvergent, falls das Integral 
        über \(f\) konvergiert.
    \end{enumerate}
\end{karte}

\begin{karte}{Majorantenkriterium Integrale}
    Seien \( f,g \) Regelfunktionen auf 
    \( [a,b), \abs{f} \leq g \). 
    Existiert \( \integral{g(x)}{a}{b} \), so existiert 
    auch \( \integral{f(x)}{a}{b} \), welches
    absolut konvergent ist.\\
    Analog für \( (a,b], (a,b) \).
\end{karte}

\begin{karte}{Integralkriterium für Konvergenz von Reihen}
    Sei \( \abb{f}{[1,\infty)}{[0,\infty)} \) monoton 
    fallend, dann konvergiert 
    \[ a_n := \sum_{k=1}^n f(k) 
    - \integral{f(x)}{1}{n+1} \]    
    und 
    \[ 0 \leq \limes{n} a_n \leq f(1) \]
    und insbesondere gilt:
    \[ \sum_{k=1}^\infty f(k) \text{ konvergiert }
    \Leftrightarrow \integral{f(x)}{1}{\infty} 
    \text{ konvergiert}. \]
    Und im Fall der Konvergenz:
    \[ 0 \leq \sum_{k=1}^\infty f(k) - \integral{f(x)}{1}{\infty} 
    \leq f(1). \]
\end{karte}

\end{document}