\documentclass[main.tex]{subfiles}
\begin{document}

\section*{Die mehrdimensionale Ableitung}

\begin{karte}{Differenzierbarkeit Komponentenfunktionen}
    Sei \( \abb{f}{(a,b)}{\R^m}, \) 
    \[f(t) = \begin{pmatrix}
        f_1(t)\\
        \vdots \\
        f_m(t)
    \end{pmatrix}. \]
    \( \abb{f_j}{(a,b)}{\R}, t_0 \in (a,b) \).
    \[ \text{Es existiert } f'(t_0) = \limesx{t}{t_0} \frac{f(t) - f(t_0)}{t - t_0} \]
    \[ \Leftrightarrow \abb{f_j}{(a,b)}{\R} 
    \text{ ist in \(t_0\) differenzierbar } \forall j=1,\ldots,m. \]
\end{karte}

\begin{karte}{Richtungsableitung}
    Sei \( \abb{f}{U}{\R^m}, U \subset \R^n \) offen, 
    \( x \in U, v \in \R^n \).
    \[ Df (x_0)[v] = 
    D_v f(x) := \limesx{t}{0} \frac{f(x + tv) - f(x)}{t} 
    =: \ddx{t} f(x + tv) \]
    heißt die Richtungsableitung von \( f \) in \( x\in U \), 
    in Richtung \( v \in \R^n \).
\end{karte}

\begin{karte}{Partielle Ableitung}
    \( \abb{f}{U}{\R^m}, U \subset \R^n \) offen, \( x\in U \).
    Die \(j\)-te partielle Ableitung von \(f\) in \(x \in U\) 
    ist gegeben durch
    \[ \partial_j f(x) := \ddxpartial{x_j} f(x) 
    = D_{e_j} f(x) = \ddx{t} f(x + t e_j) \vert_{t=0}. \]

    Wenn alle partiellen Ableitungen existieren, muss die Funktion 
    nicht unbedingt stetig sein.
\end{karte}
\begin{karte}{Rechenregeln von partiellen Ableitungen}
    Sei \( U \subset \R^n \) offen, \( x\in U \), 
    \( \abb{f,g}{U}{\R^m} \) und 
    \( \partial_j f(x), \partial_j g(x) \) existiere.
    \begin{enumerate}
        \item Linearität \( \forall \alpha, \beta \in \R: 
        \partial_j (\alpha f + \beta g)(x) 
        = \alpha \partial_j f(x) + \beta \partial_j g(x). \)
        \item \[ \partial_j f(x) = \begin{pmatrix}
            \partial_j f_1(x) \\
            \vdots \\
            \partial_j f_m(x) \\
        \end{pmatrix} \]
        \item Quotientenregel (für \( g(x) \neq 0 \))
        \[ \partial_j \left(\frac{f}{g}\right)(x) 
        = \frac{g(x) \partial_j f(x) - f(x) \partial_j g(x)}{g(x)^2} \]
        \item Kettenregel (für \( \abb{f}{U}{I \subset \R}, 
        I \) offenes Intervall, 
        \( \abb{\varphi}{I}{\R} \) differenzierbar)
        \( \partial_j (\varphi \circ f)(x) 
        = \varphi'(f(x)) \partial_j f(x) \)
        \item Produktregel: \( \abb{f}{U}{\R^m}, 
        \abb{g}{U}{\R} \), partielle Ableitung 
        existiert in \( x_0 \in U \).
        \[ \Rightarrow \partial_j(f\cdot g)(x_0) 
        = (\partial_j f)(x_0)\cdot g(x_0) + f(x_0)\cdot (\partial_j g)(x_0). \]
    \end{enumerate}
\end{karte}

\begin{karte}{Definition Ableitung}
    \( U \subset \R^n \) offen. Eine Funktion \(\abb{f}{U}{R^m}\)
    heißt differenzierbar in \(x_0 \in U\), falls ein
    \( A \in \mathscr{L}(\R^n, \R^m) \) existiert mit
    \[ \limesx{x}{x_0} \frac{f(x) -f(x_0) - A(x-x_0)}{\abs{x-x_0}} 
    = 0 \in \R^m. \]
    Mit der Substitution \( x = x_0 + h \), 
    erhält man die äquivalente Form 
    \[ \limesx{h}{0} \frac{ f(x_0 + h) - f(x_0) - A(h) }{\abs{h}} = 0. \]
    Wir nennen \( Df(x_0) = A \) die Ableitung von \(f\) ein \(x_0\). 
    \( \abb{Df(x_0)}{\R^n}{\R^m} \) ist linear.\\
    Wenn \( f \) differenzierbar ist in \( x_0 \in U \), dann ist 
    \( f \) auch stetig in \( x_0 \).
\end{karte}

\subsection*{Berechnen Ableitungen}

\begin{karte}{Jakobimatrix}
    \[ (\partial_1f(x), \dots, \partial_n f(x)) \]
    \[ = \begin{pmatrix}
        \partial_1 f_1(x) & \cdots & \partial_n f_1(x) \\
        \vdots & & \vdots \\
        \partial_1 f_m(x) & \cdots & \partial_n f_m(x)
    \end{pmatrix} \]
\end{karte}

\begin{karte}{Gradient}
    Sei \( \abb{f}{\R^n}{\R} \).
    Wir nennen \( \grad f(x) := \nabla f(x) = \begin{pmatrix}
        \partial_1 f(x) \\
        \vdots \\
        \partial_n f(x)
    \end{pmatrix} \in \R^n \) den Gradienten von \(f\) in 
    \( x \in U \).

    Ist \( \nabla f(x) = 0 \in \R^n \), so heißt \(x\) 
    kritischer Punkt.\\
    Ist \( \nabla f(x) \neq 0 \), so heißt \(x\) 
    nicht kritisch.
\end{karte}

\begin{karte}{Lin. Annäherung von Funktion durch Ableitung}
    Sei \( U \subset \R^n\) offen, \(\abb{f}{U}{\R^m}\).
    Dann gilt:
    \(f\) ist differenzierbar in \(x_0 \in U \) mit Ableitung
    \( A = Df(x_0) \)
    \( \Leftrightarrow \exists \) Funktion, \( \varepsilon = 
    \abb{\varepsilon_{x_0}}{U}{\R^m} \), stetig in \(x_0\) und
    \(\varepsilon(x_0) = 0\) mit 
    \[ f(x) = \underbrace{f(x_0) + A(x-x_0)}_{\text{affin linear}} 
    + \underbrace{\abs{x-x_0}\varepsilon(x)}_{\text{Fehler}}
    \; \forall x \in U \]
\end{karte}

\begin{karte}{Rechenregeln Ableitung}
    Linearität: \( \abb{f,g}{U}{\R^m} \) differenzierbar in 
    \(x_0 \in U \Rightarrow \alpha f + \beta g \) ist differenzierbar 
    in \(x_0 \; \forall \alpha, \beta \in \R\) \\
    \( D(\alpha f + \beta g)(x_0) = \alpha Df(x_0) + \beta Dg(x_0) \).

    Produktregel: Sei \( \abb{f}{U}{\R^m}, \abb{g}{U}{\R} \)
    differenzierbar in \(x_0\) \\
    \( \Rightarrow  fg \) differenzierbar in \(x_0\): 
    \( Dfg(x_0) = Df(x_0)g(x_0) + f(x_0)Dg(x_0) \).

    Quotientenregel \( \abb{f}{U}{\R^m}, \abb{g}{U}{\R} \)
    \[ g(x_0) \neq 0 \]
    \[ D\left(\frac{f}{g}\right)(x_0) 
    = \frac{Df(x_0) g(x_0) - Dg(x_0) f(x_0)}{g(x_0)^2}. \]
\end{karte}

\begin{karte}{Rechenregeln Ableitung: Kettenregel}
    Seien \( \abb{f}{U}{\R^m} \), 
    \( \abb{g}{V}{\R^p} \)
    mit \( U \subset \R^n, V \subset \R^m \)
    offen und \( f(U) \subset V \).
    Sind \(f\) in \(x_0\) und \(g\) in \(y_0 = f(x_0)\)
    differenzierbar, so ist auch \(\abb{g \circ f}{U}{\R^p}\)
    differenzierbar in \(x_0\) und es gilt die 
    Kettenregel 
    \[ D(g\circ f)(x_0) = D g(f(x_0))Df(x_0). \]
    Sind \( \abb{A = Dg(y_0)}{\R^m}{\R^p} \) und 
    \( B = \abb{Df(x_0)}{\R^n}{\R^m} \)
    die zugehörige Jacobimatrizen, so folgt 
    \( C := AB \)
    wobei
    \( \abb{C}{\R^n}{\R^p} \) 
    die Jacobimatrix zu \( D(f \circ g)(x_0) \) ist.
\end{karte}

\end{document}