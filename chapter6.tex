\documentclass[main.tex]{subfiles}
\begin{document}

\section*{Integrationstechniken}

\begin{karte}{Definition fast überall differenzierbar}
    Eine Funktion \( \abb{F}{I}{\C} \) ist fast überall 
    stetig differenzierbar, falls es Stammfunktion einer 
    Regelfunktion \( \abb{f}{I}{\C} \) ist.
\end{karte}

\begin{karte}{Partielle Integration}
    Sind \(\abb{u,v}{I}{\C}\) fast überall stetig 
    differenzierbar, dann ist auch 
    \( u \cdot v \) fast überall stetig differenzierbar und 
    es gilt 
    \[ \int u v' \dx = uv - \int u' v \dx. \]
    Genauer: 
    \[ \integral{uv'}{a}{b} 
    = \left[ uv \right]_a^b - \integral{u'v}{a}{b} 
    \;\forall a,b\in I. \]
\end{karte}

\begin{karte}{Substitutionsregel}
    Sei \( \abb{f}{I}{\C} \) Regelfunktion    
    und Stammfunktion \(F\) zu \(f\).
    Weiter sei \( \abb{t}{[a,b]}{I} \) 
    stetig differenzierbar und monoton.
    Dann gilt:    
    \( F \circ t \) ist Stammfunktion von 
    \( (f\circ t) \cdot t' \)
    und es gilt 
    \[ \integral{f(t(x))}{a}{b} 
    = \integralx{f(t)}{t(a)}{t(b)}{t}. \]
\end{karte}

\end{document}
