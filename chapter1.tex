\documentclass[main.tex]{subfiles}
\begin{document}

\section*{Gleichmäßige Konvergenz}

\begin{karte}{Definition gleichmäßige Konvergenz}
    Eine Folge \( \abb{f_n}{D}{\R} \) (oder \( \C \)) 
    konvergiert gleichmäßig gegen \( f: D \rightarrow \R \), 
    falls 
    \[ \limes{n} ||f_n - f ||_\infty = 0. \]
\end{karte}
\begin{karte}{Gleichmäßige Konvergenz und Stetigkeit}
    Seien \( f_n : D \rightarrow \C, n\in\N \) eine 
    stetige Funktionenfolge auf \( D \subset \R \) 
    (oder \( D = \C \)), die gleichmäßig gegen 
    \( f: D \rightarrow \C \) konvergiert, d.\ h.\ 
    \[ ||f_n - f||_\infty \rightarrow 0 
    \text{ für } n\rightarrow\infty. \]
    Dann ist \(f\) auch stetig auf \(D\).
\end{karte}

\end{document}