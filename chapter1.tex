\documentclass[main.tex]{subfiles}
\begin{document}

\section*{Gleichmäßige Konvergenz}

\begin{karte}{Definition gleichmäßige Konvergenz}
    Eine Folge \( \abb{f_n}{D}{\R} \) (oder \( \C \)) 
    konvergiert gleichmäßig gegen \( f: D \rightarrow \R \), 
    falls 
    \[ \limes{n} ||f_n - f ||_\infty = 0. \]
\end{karte}
\begin{karte}{Gleichmäßige Konvergenz und Stetigkeit}
    Seien \( f_n : D \rightarrow \C, n\in\N \) eine 
    stetige Funktionenfolge auf \( D \subset \R \) 
    (oder \( D = \C \)), die gleichmäßig gegen 
    \( f: D \rightarrow \C \) konvergiert, d.\ h.\ 
    \[ ||f_n - f||_\infty \rightarrow 0 
    \text{ für } n\rightarrow\infty. \]
    Dann ist \(f\) auch stetig auf \(D\).
\end{karte}

\section*{Taylor}

\begin{karte}{Taylorpolynom}
    Sei \( \abb{f}{(c,d)}{\R} \) \(n\)-mal differenzierbar, 
    \( a \in (c,d) \). Dann heißt 
    \[ T_n(f,a)(x) := \sum_{l=0}^n 
    \frac{f^{(l)}(a)}{l!} {(x-a)}^l \]
    das \(n\)-te Taylorpolynom von \(f\) an der Stelle \(a\) 
    und \[ R_n(f,a) := f - T_n(f,a) \] heißt Restglied.
\end{karte}

\begin{karte}{Satz von Taylor}
    Sei \( \abb{f}{I}{\R} \) \( n \)-mal differenzierbar. Dann gibt es zu
    \( a \in I \) eine Funktion \( \abb{R_n(f,a)}{I}{\R} \) mit
    \[ f = T_n(f, a) + R_n(f,a) \]
    und \( R_n(f,a) \) ist gegeben durch
    \[ R_n(f,a)(x) = \frac{1}{n!} 
    \left( f^{(n)}(\xi) - f^{(n)}(a) \right) {(x-a)}^n \]
    für ein \( \xi \) zwischen \( a \) und \( x \).
    \[ \limesx{x}{a} \frac{ R_n(f,a)(x) }{ {(x-a)}^n } = 0. \]
\end{karte}

\begin{karte}{Restglieddarstellung von Schlömilch}
    Sei \( a \in [c,d], p > 0, n \in \N, 
    \abb{f}{[c,d]}{\R}, f\in \mathcal{C}^n([c, d], \R) \) 
    und \( f^{(n+1)} \) existiere auf \( (c,d) \). Dann existiert 
    für jedes \( x \in [c,d]\setminus \set{a} \) ein \( \xi
    \in (\min(a,x), \max(a,x)) \), sodass 
    \[ R_n(f,a)(x) = \frac{f^{(n+1)}(\xi)}{ pn! } 
    (x - \xi)^{n+1-p} (x-a)^p. \]
\end{karte}

\begin{karte}{Restglieddarstellungen Lagrange, Cauchy}
    Aus Schlömilch folgt mit \( p=1 \)
    \[ R_n(f,a)(x) = \frac{f^{(n+1)}(\zeta)}{(n+1)!} (x-a)^{n+1} \text{ (Lagrange)}. \]
    
    Mit \( p = n+1 \) folgt
    \[ R_n(f,a)(x) = \frac{f^{(n+1)}(\zeta)}{n!} 
    \left( \frac{x- \zeta}{x-a} \right)^n (x-a)^{n+1} \text{ (Cauchy)}. \]
\end{karte}


\end{document}