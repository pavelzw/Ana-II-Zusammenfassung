\documentclass[main.tex]{subfiles}
\begin{document}

\section*{Extrema m. Nebenbedingungen}

\begin{karte}{Untermannigfaltigkeit}
    Sei \( 1 \leq m \leq n \), eine Menge \( M \subset \R^n \) 
    heißt \(m\)-dimensionale Untermannigfaltigkeit 
    (U-Mfk) des \( \R^n \) der Klasse \( \mathcal{C}^r \) 
    (\(r \in \N_0 \cup \set{\infty} \)), falls zu jedem 
    \( p \in M \) eine offene Menge \( U \subset \R^n \) 
    und ein \( \mathcal{C}^r \)-Diffeomorphismus
    \(\abb{\Phi}{U}{\Phi(U)}\) existiert mit 
    \[ \Phi(M \cap U) = (\R^m \times \set{0}^{n-m}) \cap \Phi(U)\]
    lokale Plättung.
\end{karte}

\begin{karte}{Untermannigfaltigkeitskriterium}
    Sei \( M \subset \R^n, m + k = n \). Dann sind äquivalent
    \begin{enumerate}
        \item \(M\) ist \(\mathcal{C}^r \) Untermannigfaltigkeit.        
        \item Niveaumengenkriterium: Zu \( p \in M \exists \) offene 
        Umgebung \( U \subset \R^n \) mit 
        \( f \in \mathcal{C}^r(U, \R^k) \), sodass
        \[ M \cap U = f^{-1}(\set{0}) \]
        und
        \[ \rk Df = k \text{ auf } U. \]
        \item Graphenkriterium zu \(p \in M \; \exists \) offene
        Umgebung \(U \times V \in \R^m \times \R^k = \R^n\) und
        \(g \in \mathcal{C}^r (U, V)\) so, dass
        nach einer Permutuation der Koordinaten gilt \(M \cap (U \times V) 
        = \set{x,g(x) : x \in U}\).
    \end{enumerate}
\end{karte}

\begin{karte}{Tangentialvektor}
    Ein Vektor \(v \in \R^n\) heißt Tangentialvektor von \(M \subset \R^n\)
    im Punkt \(p \in M\), falls
    es eine Abbildung \( \abb{\gamma}{(-\varepsilon, \varepsilon)}{M} \) mit 
    \[ \gamma(0) = p, \quad \gamma'(0) = v \]
    Schreiben 
    \( T_p M = \) Menge aller Tangentialvektoren von \(M\) im Punkt \(p\).
\end{karte}

\begin{karte}{Kern von \( Df(p) \) bei Untermannigfaltigkeiten}
    Sei \( M \in \R^n \) eine \(m\)-dimensionale 
    \( \mathcal{C}^1 \) Untermannigfaltigkeit  und \(n = m + k\).
    Ist \( p \in M \cap U = f^{-1}(\set{0}) \) für eine
    Funktion \( f \in \mathcal{C}^1(U, \R^k) \) mit 
    \( \rk Df = k \) auf \(U\),
    so gilt 
    \[ T_p M = \ker(Df(p)). \]
    Insbesondere ist \( T_p M \) ein \(m\)-dimensionaler 
    Vektorraum.
\end{karte}

\begin{karte}{Extrema mit Nebenbedingungen}
    Sei \( U \subset \R^n \) offen, \( f \in \mathcal{C}^1(U, \R^k) \) 
    (\(k\) Nebenbedingungen) und \(\phi \in \mathcal{C}^1(U, \R)\).
    Gilt dann für ein \(p \in f^{-1}(\set{0})\)
    \begin{enumerate}
        \item \( \varphi(q) \geq \varphi(p) \;\forall q \in U \) mit
        \( f(q) = 0 \) (lokales Minimum unter Nebenbedingungen)
        \item \( \rk Df (p) = k \) (maximaler Rang)
    \end{enumerate}
    Dann gibt es \( \lambda_1, \ldots, \lambda_k \in \R \) (Lagrange Multiplikatoren)
    mit 
    \[ \nabla \varphi(p) = \sum_{j=1}^k \lambda_j \nabla f_j(p). \]
\end{karte}

\end{document}