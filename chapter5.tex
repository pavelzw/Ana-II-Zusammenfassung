\documentclass[main.tex]{subfiles}
\begin{document}

\section*{HDI}

\begin{karte}{Definition Stammfunktion}
    Gegeben \( \abb{f}{I}{\C} \) (oder \( \R^n \)), 
    \(I\) Intervall    
    \( \abb{F}{I}{C} \) heißt Stammfunktion (von \(f\)), falls
    \begin{enumerate}
        \item \(F\) stetig ist.
        \item \(F\) außerhalb einer abzählbaren Teilmenge 
        \( A \) von \( I \) differenzierbar ist und 
        \[ F'(x) = f(x) \; \forall x\in I \setminus A. \]
    \end{enumerate}
\end{karte}

\begin{karte}{Hauptsatz der Differential- und Integralrechnung}
    Sei \( \abb{f}{I}{\C} \) eine Regelfunktion 
    auf Intervall \(I\). Sei \(a \in I\) fest und 
    für \( x\in I \) setzen wir 
    \( F(x) := \int_a^x f(t) \; dt. \)
    Dann gilt: 
    \begin{enumerate}
        \item \(F\) ist eine Stammfunktion zu \(f\).
        Genauer: \( F \) ist für jeden Punkt 
        \( x_0 \in I \) sowohl rechts- als auch linksseitig 
        differenzierbar mit 
        \[ F'_-(x_0) = \limesx{t}{0-} 
        \frac{F(x_0 + t) - F(x_0)}{t} 
        = f_-(x_0) = \limesx{x}{x_0-} f(x). \]
        \[ F'_+(x_0) = \limesx{t}{0+} 
        \frac{F(x_0 + t) - F(x_0)}{t} 
        = f_+(x_0) = \limesx{x}{x_0+} f(x). \]
        
        Insbesondere ist \(F\) an jeder Stetigkeitsstelle 
        \( x_0 \) von \(f\) differenzierbar mit 
        \[ F'(x_0) = f(x_0). \]
        \item Für jede beliebige Stammfunktion \(\phi \) 
        zu \(f\) auf \(I\) gilt \( \forall a,b \in I \)
        \[ \integralx{f(t)}{a}{b}{t} = \phi(b)-\phi(a) 
        =: \left[ \phi \right]^b_a = \phi \vert_a^b. \]
    \end{enumerate}
\end{karte}

\end{document}